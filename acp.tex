\documentclass[]{article}
\usepackage{lmodern}
\usepackage{amssymb,amsmath}
\usepackage{ifxetex,ifluatex}
\usepackage{fixltx2e} % provides \textsubscript
\ifnum 0\ifxetex 1\fi\ifluatex 1\fi=0 % if pdftex
  \usepackage[T1]{fontenc}
  \usepackage[utf8]{inputenc}
\else % if luatex or xelatex
  \ifxetex
    \usepackage{mathspec}
  \else
    \usepackage{fontspec}
  \fi
  \defaultfontfeatures{Ligatures=TeX,Scale=MatchLowercase}
\fi
% use upquote if available, for straight quotes in verbatim environments
\IfFileExists{upquote.sty}{\usepackage{upquote}}{}
% use microtype if available
\IfFileExists{microtype.sty}{%
\usepackage{microtype}
\UseMicrotypeSet[protrusion]{basicmath} % disable protrusion for tt fonts
}{}
\usepackage[margin=1in]{geometry}
\usepackage{hyperref}
\hypersetup{unicode=true,
            pdftitle={acp.R},
            pdfauthor={carlos},
            pdfborder={0 0 0},
            breaklinks=true}
\urlstyle{same}  % don't use monospace font for urls
\usepackage{color}
\usepackage{fancyvrb}
\newcommand{\VerbBar}{|}
\newcommand{\VERB}{\Verb[commandchars=\\\{\}]}
\DefineVerbatimEnvironment{Highlighting}{Verbatim}{commandchars=\\\{\}}
% Add ',fontsize=\small' for more characters per line
\usepackage{framed}
\definecolor{shadecolor}{RGB}{248,248,248}
\newenvironment{Shaded}{\begin{snugshade}}{\end{snugshade}}
\newcommand{\KeywordTok}[1]{\textcolor[rgb]{0.13,0.29,0.53}{\textbf{#1}}}
\newcommand{\DataTypeTok}[1]{\textcolor[rgb]{0.13,0.29,0.53}{#1}}
\newcommand{\DecValTok}[1]{\textcolor[rgb]{0.00,0.00,0.81}{#1}}
\newcommand{\BaseNTok}[1]{\textcolor[rgb]{0.00,0.00,0.81}{#1}}
\newcommand{\FloatTok}[1]{\textcolor[rgb]{0.00,0.00,0.81}{#1}}
\newcommand{\ConstantTok}[1]{\textcolor[rgb]{0.00,0.00,0.00}{#1}}
\newcommand{\CharTok}[1]{\textcolor[rgb]{0.31,0.60,0.02}{#1}}
\newcommand{\SpecialCharTok}[1]{\textcolor[rgb]{0.00,0.00,0.00}{#1}}
\newcommand{\StringTok}[1]{\textcolor[rgb]{0.31,0.60,0.02}{#1}}
\newcommand{\VerbatimStringTok}[1]{\textcolor[rgb]{0.31,0.60,0.02}{#1}}
\newcommand{\SpecialStringTok}[1]{\textcolor[rgb]{0.31,0.60,0.02}{#1}}
\newcommand{\ImportTok}[1]{#1}
\newcommand{\CommentTok}[1]{\textcolor[rgb]{0.56,0.35,0.01}{\textit{#1}}}
\newcommand{\DocumentationTok}[1]{\textcolor[rgb]{0.56,0.35,0.01}{\textbf{\textit{#1}}}}
\newcommand{\AnnotationTok}[1]{\textcolor[rgb]{0.56,0.35,0.01}{\textbf{\textit{#1}}}}
\newcommand{\CommentVarTok}[1]{\textcolor[rgb]{0.56,0.35,0.01}{\textbf{\textit{#1}}}}
\newcommand{\OtherTok}[1]{\textcolor[rgb]{0.56,0.35,0.01}{#1}}
\newcommand{\FunctionTok}[1]{\textcolor[rgb]{0.00,0.00,0.00}{#1}}
\newcommand{\VariableTok}[1]{\textcolor[rgb]{0.00,0.00,0.00}{#1}}
\newcommand{\ControlFlowTok}[1]{\textcolor[rgb]{0.13,0.29,0.53}{\textbf{#1}}}
\newcommand{\OperatorTok}[1]{\textcolor[rgb]{0.81,0.36,0.00}{\textbf{#1}}}
\newcommand{\BuiltInTok}[1]{#1}
\newcommand{\ExtensionTok}[1]{#1}
\newcommand{\PreprocessorTok}[1]{\textcolor[rgb]{0.56,0.35,0.01}{\textit{#1}}}
\newcommand{\AttributeTok}[1]{\textcolor[rgb]{0.77,0.63,0.00}{#1}}
\newcommand{\RegionMarkerTok}[1]{#1}
\newcommand{\InformationTok}[1]{\textcolor[rgb]{0.56,0.35,0.01}{\textbf{\textit{#1}}}}
\newcommand{\WarningTok}[1]{\textcolor[rgb]{0.56,0.35,0.01}{\textbf{\textit{#1}}}}
\newcommand{\AlertTok}[1]{\textcolor[rgb]{0.94,0.16,0.16}{#1}}
\newcommand{\ErrorTok}[1]{\textcolor[rgb]{0.64,0.00,0.00}{\textbf{#1}}}
\newcommand{\NormalTok}[1]{#1}
\usepackage{graphicx,grffile}
\makeatletter
\def\maxwidth{\ifdim\Gin@nat@width>\linewidth\linewidth\else\Gin@nat@width\fi}
\def\maxheight{\ifdim\Gin@nat@height>\textheight\textheight\else\Gin@nat@height\fi}
\makeatother
% Scale images if necessary, so that they will not overflow the page
% margins by default, and it is still possible to overwrite the defaults
% using explicit options in \includegraphics[width, height, ...]{}
\setkeys{Gin}{width=\maxwidth,height=\maxheight,keepaspectratio}
\IfFileExists{parskip.sty}{%
\usepackage{parskip}
}{% else
\setlength{\parindent}{0pt}
\setlength{\parskip}{6pt plus 2pt minus 1pt}
}
\setlength{\emergencystretch}{3em}  % prevent overfull lines
\providecommand{\tightlist}{%
  \setlength{\itemsep}{0pt}\setlength{\parskip}{0pt}}
\setcounter{secnumdepth}{0}
% Redefines (sub)paragraphs to behave more like sections
\ifx\paragraph\undefined\else
\let\oldparagraph\paragraph
\renewcommand{\paragraph}[1]{\oldparagraph{#1}\mbox{}}
\fi
\ifx\subparagraph\undefined\else
\let\oldsubparagraph\subparagraph
\renewcommand{\subparagraph}[1]{\oldsubparagraph{#1}\mbox{}}
\fi

%%% Use protect on footnotes to avoid problems with footnotes in titles
\let\rmarkdownfootnote\footnote%
\def\footnote{\protect\rmarkdownfootnote}

%%% Change title format to be more compact
\usepackage{titling}

% Create subtitle command for use in maketitle
\newcommand{\subtitle}[1]{
  \posttitle{
    \begin{center}\large#1\end{center}
    }
}

\setlength{\droptitle}{-2em}

  \title{acp.R}
    \pretitle{\vspace{\droptitle}\centering\huge}
  \posttitle{\par}
    \author{carlos}
    \preauthor{\centering\large\emph}
  \postauthor{\par}
      \predate{\centering\large\emph}
  \postdate{\par}
    \date{Thu Jul 26 17:37:54 2018}


\begin{document}
\maketitle

\begin{Shaded}
\begin{Highlighting}[]
\CommentTok{#ALGORITMO PCA}

\CommentTok{#Datos}
\NormalTok{X <-}\StringTok{ }\KeywordTok{c}\NormalTok{(}\FloatTok{2.5}\NormalTok{,}\FloatTok{0.5}\NormalTok{,}\FloatTok{2.2}\NormalTok{,}\FloatTok{1.9}\NormalTok{,}\FloatTok{3.1}\NormalTok{,}\FloatTok{2.3}\NormalTok{,}\DecValTok{2}\NormalTok{,}\DecValTok{1}\NormalTok{,}\FloatTok{1.5}\NormalTok{,}\FloatTok{1.1}\NormalTok{)}
\NormalTok{Y <-}\StringTok{ }\KeywordTok{c}\NormalTok{(}\FloatTok{2.4}\NormalTok{,}\FloatTok{0.7}\NormalTok{,}\FloatTok{2.9}\NormalTok{,}\FloatTok{2.2}\NormalTok{,}\DecValTok{3}\NormalTok{,}\FloatTok{2.7}\NormalTok{,}\FloatTok{1.6}\NormalTok{,}\FloatTok{1.1}\NormalTok{,}\FloatTok{1.6}\NormalTok{,}\FloatTok{0.9}\NormalTok{)}

\CommentTok{#Inicializamos el algoritmo PCA}

\CommentTok{#Paso 1, calcular las medias}
\CommentTok{#Medias}
\NormalTok{mX <-}\StringTok{ }\KeywordTok{mean}\NormalTok{(X)}
\NormalTok{mY <-}\StringTok{ }\KeywordTok{mean}\NormalTok{(Y)}
\NormalTok{mX}
\end{Highlighting}
\end{Shaded}

\begin{verbatim}
## [1] 1.81
\end{verbatim}

\begin{Shaded}
\begin{Highlighting}[]
\NormalTok{mY}
\end{Highlighting}
\end{Shaded}

\begin{verbatim}
## [1] 1.91
\end{verbatim}

\begin{Shaded}
\begin{Highlighting}[]
\CommentTok{#Paso 2}
\CommentTok{#centramos los datos, es decir, restamos la media}
\NormalTok{Data <-}\StringTok{ }\KeywordTok{cbind}\NormalTok{(X}\OperatorTok{-}\NormalTok{mX,Y}\OperatorTok{-}\NormalTok{mY)}
\NormalTok{Data}
\end{Highlighting}
\end{Shaded}

\begin{verbatim}
##        [,1]  [,2]
##  [1,]  0.69  0.49
##  [2,] -1.31 -1.21
##  [3,]  0.39  0.99
##  [4,]  0.09  0.29
##  [5,]  1.29  1.09
##  [6,]  0.49  0.79
##  [7,]  0.19 -0.31
##  [8,] -0.81 -0.81
##  [9,] -0.31 -0.31
## [10,] -0.71 -1.01
\end{verbatim}

\begin{Shaded}
\begin{Highlighting}[]
\CommentTok{#Matriz de convarianza con los datos centrados}
\NormalTok{covData <-}\StringTok{ }\NormalTok{(}\DecValTok{1}\OperatorTok{/}\DecValTok{10}\NormalTok{)}\OperatorTok{*}\KeywordTok{t}\NormalTok{(Data)}\OperatorTok\NormalTok{Data}
\NormalTok{covData}
\end{Highlighting}
\end{Shaded}

\begin{verbatim}
##        [,1]   [,2]
## [1,] 0.5549 0.5539
## [2,] 0.5539 0.6449
\end{verbatim}

\begin{Shaded}
\begin{Highlighting}[]
\CommentTok{#Paso 3}
\CommentTok{#Calculamos los vectores y valores propios de la matriz de covarianza}
\NormalTok{pcas <-}\StringTok{ }\KeywordTok{eigen}\NormalTok{(covData)}
\NormalTok{pcas}\OperatorTok{$}\NormalTok{vectors}
\end{Highlighting}
\end{Shaded}

\begin{verbatim}
##           [,1]       [,2]
## [1,] 0.6778734 -0.7351787
## [2,] 0.7351787  0.6778734
\end{verbatim}

\begin{Shaded}
\begin{Highlighting}[]
\NormalTok{pcas}
\end{Highlighting}
\end{Shaded}

\begin{verbatim}
## eigen() decomposition
## $values
## [1] 1.15562494 0.04417506
## 
## $vectors
##           [,1]       [,2]
## [1,] 0.6778734 -0.7351787
## [2,] 0.7351787  0.6778734
\end{verbatim}


\end{document}
